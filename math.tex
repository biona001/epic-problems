\section{Math}

\begin{problembox}{}{}
Let $\bX = \R^{n \times n}$ random matrix. Show that probability that $\det(\bX) = 0$ is $0$. That is, almost all $n \times n$ random matrices are invertible. 
\end{problembox}

\begin{problembox}{When middle school algebra surpase college calculus}{}
For $n \in \R$, prove that the following optimization problem
\begin{align*}
	\max &\quad xy\\
	\text{s.t.} & \quad x + y = n.
\end{align*}
has optimal point $x = y = \frac{n}{2}$. Then show that, with the additional constraint that $x, y, n \in \Z$, the solution is achieved by $x = \ceil{n/2}, y = \floor{n/2}$. 
\end{problembox}

\begin{proof}
Since $y = n - x$, the problem is equivalent to maximizing $x(n - x)$ with no constraint. Completing the square, we have
\begin{align*}
	x(n - x) = -(x^2 - nx) = -\left(x - \frac{n}{2}\right)^2 + \frac{n}{4}.
\end{align*}
Since $n$ is fixed, the objectice is maximized when $-\left(x - \frac{n}{2}\right)^2 = 0 \iff x = n / 2 = y.$ If we seek integer solutions, minimizing $x - n/2$ is achieved by rounding $n/2$ to the nearest integer. Thus $y$ is just $n - \ceil{n / 2} = \floor{n/2}$. \\
\\
\textbf{Note to self:} the intuitive method of differentiation natural to all calculus students fails for the integer case, whereas completing the square method natural to middle school students is straightforward. 
\end{proof}

\begin{problembox}{}{}
Continuing the previous problem, for $n \in \R$, prove that the following optimization problem
\begin{align*}
	\max &\quad x_1 x_2 \cdots x_m\\
	\text{s.t.} & \quad \sum_{i=1}^m x_i = n.
\end{align*}
has optimal point $x_i = \frac{n}{m}$. What would the solution look like with the additional constraint $x_i, n \in \Z$? 
\end{problembox}

\begin{problembox}{}{}
Given a line of length $l$, show that the maximum area it can enclose is achieved by a circle of radius $\frac{l}{2\pi}.$
\end{problembox}


\begin{problembox}{}{}
Suppose matrix $\bM$ is orthogonal and upper triangular, show that $\bM$ is diagonal with $\pm1$ on the diagonal.
\end{problembox}

\begin{proof}
Write $M = [\bv_1, ..., \bv_n]$ where each $\bv_i$ are column vectors with $n$ terms. Then the upper triangularity of $M$ implies that 
\begin{align*}
\bv_1 = 
\begin{bmatrix}
    a_{11} \\
    0 \\ 
    \vdots \\
	0 \\
\end{bmatrix}
, \quad \bv_2 = 
\begin{bmatrix}
    a_{12} \\
    a_{22} \\
    0 \\ 
	\vdots \\
	0 \\
\end{bmatrix}
, \quad \bv_3 = 
\begin{bmatrix}
    a_{13} \\
    a_{23} \\
    a_{33} \\ 
    0 \\
	\vdots \\
	0 \\
\end{bmatrix}
... \text{and so on.}
\end{align*}

Since $\bM$ is orthogonal, $\bv_1^t\bv_1 = 1 \iff a_{11} = \pm1$. Furthermore, orthogonality implies that $\bv_1^t \bv_2 = 0 \iff a_{12} = 0 \iff v_2 = [0, a_{22}, 0, ..., 0]^t$. Again $a_{22} = \pm1$ since $\bv_2^t\bv_2 = 1$. The result follows by induction on $n$. 
\end{proof}

\begin{problembox}{\hfill {\small \cite[Exercise 8.23]{lange2010numerical}}}{}
Use the Gerschgorin circle theorem to estimate eigenvalues of the following matrix:
\[ 
\begin{bmatrix}
    4 & 0.2 & -0.1 & 0.1\\
    0.2 & -1 & -0.1 & 0.05\\
    -0.1 & -0.1 & 3 & 0.1\\
    0.1 & 0.05 & 0.1 & -3
\end{bmatrix}
\]
\end{problembox}

\begin{proof}
Since the matrix is symmetric, checking along the rows or along the columns would yield the same invervals. The eigenvalues lie within the four disks as follows:
$$D(4, 0.4) = [3.6, 4.4], \quad D(-1, 0.35) = [-1.35, -0.65]$$
$$ D(3, 0.3) = [2.7, 3.3], \quad D(-3, 0.25) = [-3.25, -2.75].$$
The actual eigenvalues are $4.0198, -3.00433, 2.99365$, and $-1.00911$, which indeed lies within our estimated invervals. \\
\\
\textbf{Note to self:} From this it seems like the eigenvalue can be better estimated by $a_{ii} + \sum_{i \neq j} a_{ij} $, so the disk interval can be decreased by half. For instance, $ D(4,0.4) = [4, 4.4]$ since $0.2-0.1+0.1 = 0.1 =$ positive, so we can exclude the interval [3.6, 4].
\end{proof}
