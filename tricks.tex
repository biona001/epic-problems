\section{Useful Tricks and Identities}

\begin{problembox}{\hfill {\small \cite[Chapter 3.4]{dobson2008introduction}}}{}
Let $\bX \in \R^{n \times p}$, $\lambda_i \in \R$, and $\bx_i^T \in \R^p$ be a row of $\bX$. Show that
\begin{align*}
	\sum_{i=1}^n \lambda_i \bx_i\bx_i^T = \bX^T
	\begin{bmatrix}
		\lambda_1 & & \bzero \\
		& \ddots & \\
		\bzero & & \lambda_n
	\end{bmatrix} \bX
\end{align*}
\end{problembox}

\begin{proof}
This is a definition problem. By definition we have 
\begin{align*}
	\bX^T\bX = 
	\begin{bmatrix}
		\vrule & & \vrule\\
		\bx_1 & \horzbar & \bx_n\\
		\vrule & & \vrule
	\end{bmatrix}
	\begin{bmatrix}
		\horzbar & \bx_1^T & \horzbar\\
		& \vrule & \\
		\horzbar & \bx_n^T & \horzbar
	\end{bmatrix}
	\equiv
	\begin{bmatrix}
	c_{11} & \cdots & c_{ij}\\
	\vdots & & \vdots \\
	& & c_{nn}
	\end{bmatrix}		
\end{align*}
Therefore $c_{11} = x_{11}x_{11} + x_{21}x_{21} + ... + x_{n1}x_{n1}.$ Similarly,
\begin{align*}
	\sum_{i=1}^n \bx_i\bx_i^T
	= 
	\begin{bmatrix}
		d_{11} & \cdots & d_{ij}\\
	    \vdots & & \vdots \\
	    & & d_{nn}
	\end{bmatrix}
	\iff d_{11} = \left(\bx_1\bx_1^T\right)_{11} + \left(\bx_2\bx_2^T\right)_{11}... + \left(\bx_n\bx_n^T\right)_{11} = c_{11}.
\end{align*}
Therefore the entries match up judiciously. 
\end{proof}

\begin{problembox}{Exact 2nd order Taylor's expansion}{}
Suppose $f \in C^{2}(\R)$. Show that there exists $y \in (x_0, x)$ such that:
\begin{align*}
f(x) = f(x_0) + f'(x_0)(x - x_0) + \frac{1}{2} f''(y)(x - x_0)^2.
\end{align*}
This motivates the quadratic upper bound principle, which is used ubiquitously in MM algorithms.
\end{problembox}

\begin{proof}
Applying fundamental theorem of calculus twice, we have
\begin{align*}
	f(x) 
	&= f(x_0) + \int_{x_0}^x f'(x_1) dx_1\\
	&=f(x_0) + \int_{x_0}^x \left(f'(x_0) + \int_{x_0}^{x_1} f''(x_2)dx_2 \right)dx_1\\
	&= f(x_0) + f'(x_0)(x - x_0) + \int_{x_0}^x\int_{x_0}^{x_1}f''(x_2)dx_2dx_1.
\end{align*}
By mean value theorem, there exists $y \in (x_0, x_1)$ such that $\int_{x_0}^{x_1}f''(x_2)dx_2 = f''(y)(x_1 - x_0).$ Thus
\begin{align*}
	f(x)
	&= f(x_0) + f'(x_0)(x - x_0) +\int_{x_0}^x f''(y)(x_1 - x_0)dx_1\\
	&= f(x_0) + f'(x_0)(x - x_0) + \frac{1}{2}f''(y)(x - x_0)^2.
\end{align*}
To complete the proof, note that $x_1 \in (x_0, x)$, so $y \in (x_0, x)$.
\end{proof}

\begin{problembox}{Clever use of Cauchy-Schwarz \hfill {\small \cite[Exercise 1.4.18]{lange2016mm}}}{}
Prove the majorization
\begin{equation*}
\begin{split}
(x + y - z)^2 \leq -(x_n + y_n - z_n)^2 + 2(x_n + y_n-z_n)(x+y-z) \\
+ 3[(x-x_n)^2 + (y-y_n)^2 + (z-z_n)^2]
\end{split}
\end{equation*}
which separtes the variables $x, y, $ and $z$. In examples 1.3.6 and 1.3.7 this would facilitate penalizing parameter curvature rather than changes in parameter values. 
\end{problembox}

\begin{proof}
First, move the first two terms on the right to the left: 
$$(x + y - z)^2 - 2(x_n + y_n-z_n)(x+y-z) + (x_n + y_n - z_n)^2 \leq 3[(x-x_n)^2 + (y-y_n)^2 + (z-z_n)^2]$$
The left can be factored cleanly as
\begin{align*}
\big( (x + y - z) - (x_n + y_n - z_n)\big)^2 \leq 3[(x-x_n)^2 + (y-y_n)^2 + (z-z_n)^2]
\end{align*}
\begin{align*}
\iff (a + b + c)^2 \leq 3a^2 + 3b^2 + 3c^2
\end{align*}
where $a = x - x_n, b = y - y_n, c = z - z_n.$ Now define $v = (1, 1, 1), u = (a, b, c).$ By Cauchy-Schwarz, we obtain the desired result:
$$(a + b + c)^2 \leq 3 (a^2 + b^2 + c^2).$$
\end{proof}
